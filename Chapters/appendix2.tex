%!TEX root = ../template.tex
%%%%%%%%%%%%%%%%%%%%%%%%%%%%%%%%%%%%%%%%%%%%%%%%%%%%%%%%%%%%%%%%%%%%
%% appendix2.tex
%% NOVA thesis document file
%%
%% Chapter with OpenCV description text
%%%%%%%%%%%%%%%%%%%%%%%%%%%%%%%%%%%%%%%%%%%%%%%%%%%%%%%%%%%%%%%%%%%%
\chapter{OpenCV}
\label{app:opencv_functions}

\section{Introduction}
\label{sec:opencv_appendix_introduction}

OpenCV (Open Source Computer Vision Library) is an open source computer vision and machine learning software library. Within its 20 years of history, the library has accumulated more than 2500 algorithms, both classic and state-of-the-art. These algorithms provide a huge range of applications like facial detection, object identification, classification of human actions in video, tracking moving objects, producing 3D point clouds from stereo cameras, scenery recognition with overlay markers in augmented reality, etc.

OpenCV has a huge community of users and it is used extensively in big companies and startups alike, within research groups and by governmental bodies.

It is written natively in C++ and has a templated interface that works seamlessly with STL containers. It also has Python, Java and MATLAB interfaces and supports Windows, Linux, Android and Mac OS. OpenCV leans mostly towards real-time vision applications.

The version used on this thesis was 3.2.0.

% section opencv_appendix_introduction

\section{Used Functions}
\label{sec:opencv_appendix_used_functions}

This section presents the documentation of the OpenCV functions used in the algorithms developed on this thesis.

\subsection*{cvtColor}
\label{subsec:opencv_appendix_used_functions_cvtcolor}

\textit{\textbf{void} cv::cvtColor (
        \textbf{InputArray} src,
		\textbf{OutputArray}  	dst,
		\textbf{int}  	code,
		\textbf{int}  	dstCn = 0)}\\
		
The function converts an input image from one color space to another. In case of a transformation to-from RGB color space, the order of the channels should be specified explicitly (RGB or BGR). Note that the default color format in OpenCV is often referred to as RGB but it is actually BGR (the bytes are reversed). So the first byte in a standard (24-bit) color image will be an 8-bit Blue component, the second byte will be Green, and the third byte will be Red. The fourth, fifth, and sixth bytes would then be the second pixel (Blue, then Green, then Red), and so on.

The conventional ranges for R, G, and B channel values are:

\begin{itemize}
    \item 0 to 255 for CV\_8U images;
    \item 0 to 65535 for CV\_16U images;
    \item 0 to 1 for CV\_32F images.
\end{itemize}

\textbf{Parameters}
\begin{itemize}
    \item \textbf{src} input image: 8-bit unsigned, 16-bit unsigned ( CV\_16UC... ), or single-precision floating-point.
    \item \textbf{dst} output image of the same size and depth as src.
    \item \textbf{code} color space conversion code (see cv::ColorConversionCodes).
    \item \textbf{dstCn} number of channels in the destination image; if the parameter is 0, the number of the channels is derived automatically from src and code.
\end{itemize}

\begin{adjustbox}{width=1.2\textwidth,center=\textwidth}
\textbf{Source:} \url{https://docs.opencv.org/3.2.0/d7/d1b/group__imgproc__misc.html#ga397ae87e1288a81d2363b61574eb8cab}
\end{adjustbox}

\noindent\rule{\textwidth}{0.5pt}
		
% subsection opencv_appendix_used_functions_cvtcolor

\subsection*{resize}
\label{subsec:opencv_appendix_used_functions_resize}

\textit{\textbf{void} cv::resize (
        \textbf{InputArray} src,
		\textbf{OutputArray}  	dst,
		\textbf{Size}  	dsize,
		\textbf{double}  	fx = 0,
		\textbf{double}  	fy = 0,
		\textbf{int}  	interpolation = INTER\_LINEAR)}\\
		
The function resize resizes the image src down to or up to the specified size. Note that the initial dst type or size are not taken into account. Instead, the size and type are derived from the src, dsize, fx, and fy.

To shrink an image, it will generally look best with cv::INTER\_AREA interpolation, whereas to enlarge an image, it will generally look best with cv::INTER\_CUBIC (slow) or cv::INTER\_LINEAR (faster but still looks OK).\\

\textbf{Parameters}
\begin{itemize}
    \item \textbf{src} input image.
    \item \textbf{dst} output image; it has the size dsize (when it is non-zero) or the size computed from src.size(), fx, and fy; the type of dst is the same as of src.
    \item \textbf{dsize} output image size; if it equals zero, it is computed as $$dsize = Size(round(fx*src.cols), round(fy*src.rows))$$
    Either dsize or both fx and fy must be non-zero.
    \item \textbf{fx} scale factor along the horizontal axis; when it equals 0, it is computed as $$(double)dsize.width/src.cols$$
    \item \textbf{fy}	scale factor along the vertical axis; when it equals 0, it is computed as $$(double)dsize.height/src.rows$$
    \item \textbf{interpolation}	interpolation method, see cv::InterpolationFlags
\end{itemize}

\begin{adjustbox}{width=1.2\textwidth,center=\textwidth}
\textbf{Source:} \url{https://docs.opencv.org/3.2.0/da/d54/group__imgproc__transform.html#ga47a974309e9102f5f08231edc7e7529d}
\end{adjustbox}

\noindent\rule{\textwidth}{0.5pt}

% subsection opencv_appendix_used_functions_resize

\subsection*{threshold}
\label{subsec:opencv_appendix_used_functions_threshold}

\textit{\textbf{double} cv::threshold (
        \textbf{InputArray} src,
		\textbf{OutputArray}  	dst,
		\textbf{double}  	thresh,
		\textbf{double}  	maxval,
		\textbf{int}  	type)}\\

The function applies fixed-level thresholding to a single-channel array. The function is typically used to get a bi-level (binary) image out of a grayscale image ( cv::compare could be also used for this purpose) or for removing a noise, that is, filtering out pixels with too small or too large values. There are several types of thresholding supported by the function. They are determined by type parameter.

Also, the special values cv::THRESH\_OTSU or cv::THRESH\_TRIANGLE may be combined with one of the above values. In these cases, the function determines the optimal threshold value using the Otsu's or Triangle algorithm and uses it instead of the specified thresh . The function returns the computed threshold value. Currently, the Otsu's and Triangle methods are implemented only for 8-bit images.\\

\textbf{Parameters}
\begin{itemize}
    \item \textbf{src} input array (single-channel, 8-bit or 32-bit floating point).
    \item \textbf{dst} output array of the same size and type as src.
    \item \textbf{thresh} threshold value.
    \item \textbf{maxval} maximum value to use with the THRESH\_BINARY and THRESH\_BINARY\_INV thresholding types.
    \item \textbf{type}	thresholding type (see the cv::ThresholdTypes).
\end{itemize}

\begin{adjustbox}{width=1.2\textwidth,center=\textwidth}
\textbf{Source:} \url{https://docs.opencv.org/3.2.0/d7/d1b/group__imgproc__misc.html#gae8a4a146d1ca78c626a53577199e9c57}
\end{adjustbox}

\noindent\rule{\textwidth}{0.5pt}

% subsection opencv_appendix_used_functions_threshold

% section opencv_appendix_used_functions
