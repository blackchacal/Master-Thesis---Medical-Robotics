%!TEX root = ../template.tex
%%%%%%%%%%%%%%%%%%%%%%%%%%%%%%%%%%%%%%%%%%%%%%%%%%%%%%%%%%%%%%%%%%%%
%% abstrac-pt.tex
%% NOVA thesis document file
%%
%% Abstract in Portuguese
%%%%%%%%%%%%%%%%%%%%%%%%%%%%%%%%%%%%%%%%%%%%%%%%%%%%%%%%%%%%%%%%%%%%

As queimaduras ainda representam um problema médico da actualidade, tendo um impacto mundial considerável no número de mortos e grau de incapacidade. Umas das promessas futuras de tratamento de queimaduras extensivas consiste na utilização da bioimpressão 3D em vez de enxertos de pele. Os tecidos impressos irão substituir ou auxiliar o processo de regeneração das zonas afectadas.

A bioimpressão 3D tem um elevado potencial mas ainda tem que ultrapassar alguns desafios para poder ser usada de forma mais transversal na indústria médica. Uma das limitações é a bioimpressão in situ. A maioria das biomimpressoras existentes imprimem no interior do seu volume de impressão, depois transferem o resultado para um bioreactor para maturação e finalmente para o local desejado. Este processo leva tempo e limita o escalonamento do processo. Ao imprimir in situ, a maturação pode ser feita no destino final reduzindo o tempo de todo o processo.

Este trabalho, usando o problema das queimaduras como caso de uso, propõe-se a apresentar uma possível solução para a limitação da bioimpressão externa in situ. O sistema proposto usa um braço robótico como sistema de posicionamento e uma câmara de profundidade para detecção e avaliação das queimaduras. Durante o desenvolvimento do sistema foi usado um algoritmo e modelo de queimadura simplificados; três caminhos diferentes de preenchimento da ferida foram criados; e um controlador de impedância foi usado para o controlo do braço robótico. O sistema foi capaz de detectar o modelo da queimadura, planear uma trajectória de preenchimento e executá-la autonomamente após aprovação. Não houve impressão efectiva. O sistema foi testado em ambiente de simulação apresentando bons resultados em geral, mas com limitações em alguns passos.

Embora tenham surgido várias limitações, este trabalho mostra que é possível ter um sistema semi-autónomo a fazer bioimpressão 3D directamente sobre queimaduras.


% Palavras-chave do resumo em Português
\begin{keywords}
Bioimpressão 3D, Bioimpressão In Situ, Robótica Colaborativa, Robótica Médica, Segmentação de Feridas, Planeamento de Trajectórias, Controlo de Impedância, Automação.
\end{keywords}
% to add an extra black line
