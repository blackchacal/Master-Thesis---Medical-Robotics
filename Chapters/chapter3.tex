%!TEX root = ../template.tex
%%%%%%%%%%%%%%%%%%%%%%%%%%%%%%%%%%%%%%%%%%%%%%%%%%%%%%%%%%%%%%%%%%%%
%% chapter3.tex
%% NOVA thesis document file
%%
%% Chapter with the literature review
%%%%%%%%%%%%%%%%%%%%%%%%%%%%%%%%%%%%%%%%%%%%%%%%%%%%%%%%%%%%%%%%%%%%

\chapter{Literature Review}
\label{cha:literature_review}

3D Bioprinting is a promising technology to solve the increasing demand for organ donation, full-thickness skin replacements and in vitro tissue models. However, it still faces a few limitations throughout its three stages of the bioprinting process. This work intends to propose a solution to some of the problems associated with the bioprinting stage, in particular, the way the bioprinter works.

This chapter will focus on presenting the state-of-the-art on the positioning systems used for 3D bioprinting (both external and internal), and how that relates to resolution, speed and in situ bioprinting. It will also focus on the state-of-the-art of using vision systems to increase the system's autonomy by detecting the printing site or evaluating the printing process in real-time. Finally, it will discuss the state-of-the-art on path generation for 3D bioprinting.

% ==========================
% = 3D Positioning Systems =
% ==========================

\section{3D Positioning systems}
\label{sec:3d_positioning_systems}

Three dimensional bioprinters come in different shapes and sizes. There exists commercial bioprinters, bioprinters developed solely for research purposes, and open source bioprinters. Although all different, there is something that unite most of the existing systems. Most of them use a gantry system as the positioning system, i.e., use a Cartesian system. This only allows printing to occur horizontally and mostly on planar surfaces. Despite their limitations to work on non-planar surfaces, there were some success cases reported {\color{red} Referencias}.

In 2016, \citeauthor{Ozbolat2017_evaluation_bioprinter_tech}, made a comprehensive review of existing bioprinter technologies. They mentioned ten capabilities of an ideal bioprinter. These are high \gls{dof} in motion; high resolution and accuracy; high speed motion; the ability to dispense various bioink solutions simultaneously; ease of use; compact size; ease of sterilization; full-automation capability; affordability; and versatility. None of the bioprinters assessed had all these capabilities.

Altogether, fifty-eight bioprinters were evaluated, 17 \gls{ebb}, 33 \gls{dbb} and 8 \gls{lbb} systems. \gls{ebb} systems were all commercial. \gls{dbb} systems were 7 commercial and the rest non-commercial systems. \gls{lbb} systems were all non-commercial. No commercial \gls{lbb} system was available on the market at that time.

Of all bioprinters analysed only one had more than 3 \gls{dof}. This bioprinter was BioAssemblyBot\textregistered which is a commercial system developed by US-based company Advanced Solutions, Inc. It uses a custom-made 6 \gls{dof} robotic arm as its 3D positioning system \cite{Advanced2020_bioassemblybot}. Another interesting aspect is that none of the bioprinters assessed were used for in situ bioprinting. All of them bioprint the construct inside their building volume, which is later transferred to its final destination.

Using multi-axis robotic manipulators for 3D printing is a new trend in additive manufacturing \cite{Urhal2019_robot_assisted_additive_manufacturing_review}. Having positioning systems with more than 3 \gls{dof} has several advantages to the printing process. The most obvious advantage is the ability to print onto non-planar surfaces. In the case of bioprinting, for systems to be able to print in situ they must be able to handle the natural curves of the human body. On most of the body surface only an area of a few squared millimeters could be approximated to a planar surface {\color{red} Referencias}. This means the body is a highly non-planar surface.

Another advantage is dealing with patient positioning. If a system has to bioprint skin tissue in situ to an extensive burn wound, the system may need to adapt to the patient and not the opposite. This means it must adjust to the patient position and relative position of the bioprinting site. If the bioprinting site has an inclined orientation, a 3 \gls{dof} system will not be able to change its orientation and correctly do the printing work.

One last advantage to mention, is considering the possibility of internal bioprinting, i.e., the capability to bioprint directly onto an internal structure like an organ or joint during a surgical procedure. The body's internal environment is highly populated with different structures. This means many obstacles to position correctly the bioprinter deposition system. Having higher degrees of freedom helps the system to evade obstacles more efficiently.

Some research efforts tried to use the capabilities of robotic manipulators to enhance the bioprinting process, both in vitro and in situ (internally and externally).

In 2018, a conference paper by \citeauthor{Jafari2018_robot_system_automated_wound_filling}\cite{Jafari2018_robot_system_automated_wound_filling}, presented the use of a robotic manipulator to do automated wound filling. The system consisted on Staubli TX90 6 \gls{dof} industrial robotic manipulator paired with a camera to detect the wounds and plan the printing path.

A group from UK, \citeauthor{Lipskas2019_robotic_assisted_3dbioprint_repairing_bone_cartilage}\cite{Lipskas2019_robotic_assisted_3dbioprint_repairing_bone_cartilage}, developed a robotic-assisted 3D bioprinting system for bone and cartilage repairing using a minimally invasive approach. They built a custom robotic manipulator specially designed to be used on a \gls{mis} like arthroscopy. The manipulator has 6 \gls{dof}, with some range limits on the end-effector orientation. The system has three different end-effector, one for bone machining, another for surface registration and another for bioprinting.

Another example, focused on \gls{mis}, is the Endo AM system which is an endoscopic additive manufacturing tool \cite{Simeunovic2019_endoscopic_additive_manufacturing}. Conceived by \citeauthor{Simeunovic2019_endoscopic_additive_manufacturing}, the system consists on a 9 \gls{dof} robotic manipulator and an extruder. It is based on the Da Vinci Xi system. \\

As stated before, having more than 3 \gls{dof} is useful for in situ bioprinting. A review of in situ bioprinting studies from \citeauthor{Singh2020a_in_situ_bioprinting}\cite{Singh2020a_in_situ_bioprinting} presents a few approaches that handle the difficulties posed by the body contour, without multi-axis robots. The review focuses on robotic arm solutions and handheld solutions. Two examples for skin bioprinting are worth mentioning.

The first example is a handheld system developed by \citeauthor{Hakimi2018_handheld_skin_printer}\cite{Hakimi2018_handheld_skin_printer}. The system allows real-time bioprinting of skin-compatible biomaterials sheets on to the skin. Being handheld it can be controlled by the user which will take care of its positioning. Because it prints sheets the positioning does not require high precision and it can done by a person.

The other example, developed by \citeauthor{Albanna2019_in_situ_bioprinting_mobile_gantry}\cite{Albanna2019_in_situ_bioprinting_mobile_gantry}, is a robotic system that uses a gantry system fixed on a mobile platform to bioprint skin constructs. The system is uncapable of dealing with inclined surfaces but the mobile plaform allows it to reach most parts of the patient and print in situ. A positive aspect of this system is that it can deal with extensive excisional full-thickness wounds.

% section 3d_positioning_systems

\section{Computer Vision and image processing}
\label{sec:computer_vision_and_image_processing}

Falar do processamento de imagem para:
- Segmentação de feridas
- Quantificação da TBSA

% subsection computer_vision_and_image_processing

\section{Collaborative robotics control architectures}
\label{sec:collaborative_robotics_control_architectures}

Contextualizar o papel da robótica na cirurgia com a apresentação de vários exemplos de sucesso e apresentação das vantagens destes sistemas no contexto cirúrgico.

% subsection collaborative_robotics_control_architectures