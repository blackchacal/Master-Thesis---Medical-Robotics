%!TEX root = ../template.tex
%%%%%%%%%%%%%%%%%%%%%%%%%%%%%%%%%%%%%%%%%%%%%%%%%%%%%%%%%%%%%%%%%%%
%% chapter1.tex
%% NOVA thesis document file
%%
%% Chapter with introduciton
%%%%%%%%%%%%%%%%%%%%%%%%%%%%%%%%%%%%%%%%%%%%%%%%%%%%%%%%%%%%%%%%%%%
\newcommand{\novathesis}{\emph{novathesis}}
\newcommand{\novathesisclass}{\texttt{novathesis.cls}}


\chapter{Introduction}
\label{cha:introduction}

\begin{quotation}
  \itshape
  This work is licensed under the Creative Commons Attribution-NonCommercial~4.0 International License.
  To view a copy of this license, visit \url{http://creativecommons.org/licenses/by-nc/4.0/}.
\end{quotation}

\section{The reality of burn injury} % (fold)
\label{sec:the_reality_of_burn_injury}

Burn injuries are still a prevalent problem in today's society, claiming the lives of about 17 people worldwide, every hour, by 2016 \cite{GHE2016_xls}. Although it represents a small percentage of global deaths, burn wounds have a significant impact on morbidity and mortality and are considered the worst type of traumatic injury {\color{red}(needs ref)}. They also represent a considerable societal cost associated with the expenses for medical care and physical and psychological rehabilitation. When not leading to death, they commonly leave the person "with lifelong disabilities and disfigurements, often with resulting stigma and rejection." \cite{who2011_sucess_stories}

According to the World Health Organization (WHO) \cite{who2011_sucess_stories}, the occurrence and outcome severity of burn injury are higher on Low and Middle Income Countries (LMIC) than on High Income Countries (HIC), with 95\% of annual casualties happening on LMIC. Regarding gender distribution, females represent the majority of the victims. The most vulnerable groups are children and elders.

Burn injuries are highly preventable. The success of prevention measures applied on HIC have considerably reduced the number of casualties  

Burn injuries can be classified according to various criteria, with location, extension or percentage of Total Body Surface Area (\%TBSA), aetiology and depth, being the most common. In HIC the most common aetiology are scalds associated with food preparation or consumption and bathing.

One of the main causes of death for burned patients is sepsis. The fragility of the immune system, ..., renders the patients more prone to infectious diseases. Several types of microbial agents have been found on burn wounds, from bacteria to virus and fungi. If not detected and treated early, these infections can significantly impact morbidity and eventually lead to death.

In 2016 there were around 150 thousand deaths associated with fire, heat and hot substances, represent 0.3 \% of global deaths \cite{GHE2016_xls}.

% section the_reality_of_burn_injury (end)

\section{Goals} % (fold)
\label{sec:goals}

The present thesis proposes a new system for autonomous topical application of antimicrobial drugs on burn wounds. The main goal of the system is to reduce the time necessary to apply the drugs on extensive wounds, freeing the personnel for other important treatment tasks. This goal encompasses several specific goals, to mention:
\begin{enumerate}
    \item Detection and segmentation of burn wounds;
    \item Definition of wound perimeter and localisation;
    \item Trajectory definition;
    \item Robot manipulator control for trajectory execution;
\end{enumerate}
The broader goal of this work is to prove that there exist other valid use cases, aside from the existing ones, for the use of robotics within surgery and treatment.

% section goals

\section{Development of complementary technologies and practices (state-of-the-art)} % (fold)
\label{sec:development_of_complementary_technologies_and_practices}

In parallel with the development of medical burn wound care, other areas of science and technology are evolving which can contribute to the improvement of treatment outcome. Three areas are emphasised on this document because they can complement each other to open new roads for treatment or intervention protocols.

The first area, and more directly related with treatment itself, is tissue engineering. The development of this field as brought new perspectives on skin regeneration and healing, which holds promise of significant improvement on burn wound regeneration and scar reduction. 

Secondly, comes computer vision and image processing. The development of faster and more robust algorithms has given computers the capability of automating pattern recognition, allowing them to count cells, detect a person's face, among other feats. Within the context of burn wound care, they can be use to detect the number and size of wounds and estimate the \%TBSA. 

Lastly, collaborative robotics, with new manipulators and control architectures, allows the automation of monotonous and repetitive tasks, leaving the personnel free for more important and demanding tasks. Additionally, it brings a level of precision and accuracy unmatched by humans. Collaborative robotics brings another card to the table by permitting the coexistence of humans and robots on the same workspace.

\bigskip
Now, each area will be developed further with the respective state-of-the-art. A broader view will be used for the sake of context and to support potential future applications. A more focused and detailed view will frame the proposed goals.

\subsection{Tissue engineering}
\label{subsec:tissue_engineering}

Abordar os novos desenvolvimentos no contexto da engenharia de células e tecidos e regeneração da pele:
- Reconstrução e impressão de orgãos
- Pele artificial
- Cultura de células para regeneração da pele

% subsection tissue_engineering

\subsection{Computer Vision and image processing}
\label{subsec:computer_vision_and_image_processing}

Falar do processamento de imagem para:
- Segmentação de feridas
- Quantificação da TBSA

% subsection computer_vision_and_image_processing

\subsection{Collaborative robotics control architectures}
\label{subsec:collaborative_robotics_control_architectures}

Contextualizar o papel da robótica na cirurgia com a apresentação de vários exemplos de sucesso e apresentação das vantagens destes sistemas no contexto cirúrgico.

% subsection collaborative_robotics_control_architectures

% section development_of_complementary_technologies_and_practices (end)

\section{Proposed System Architecture} % (fold)
\label{sec:proposed_system_architecture}

% section proposed_system_architecture

\section{Document structure}
\label{sec:document_structure}

% section document_structure