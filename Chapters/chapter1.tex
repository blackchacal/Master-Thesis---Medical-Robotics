%!TEX root = ../template.tex
%%%%%%%%%%%%%%%%%%%%%%%%%%%%%%%%%%%%%%%%%%%%%%%%%%%%%%%%%%%%%%%%%%%
%% chapter1.tex
%% NOVA thesis document file
%%
%% Chapter with introduction
%%%%%%%%%%%%%%%%%%%%%%%%%%%%%%%%%%%%%%%%%%%%%%%%%%%%%%%%%%%%%%%%%%%
\newcommand{\novathesis}{\emph{novathesis}}
\newcommand{\novathesisclass}{\texttt{novathesis.cls}}


\chapter{Introduction}
\label{cha:introduction}

\begin{quotation}
\begin{flushright}
\itshape
«“Begin at the beginning," the King said, very gravely, \\"and go on till you come to the end: then stop.”»\\
\textbf{- Lewis Carroll, Alice in Wonderland}
\end{flushright}
\end{quotation}

Health is one of the pillars of well-being. Although very precious, health is volatile and subject to constant disturbances imposed by disease and injury. One such type of disturbance is caused by burn injuries. Burn injury still has an expression worldwide as a cause of death and contribute heavily to morbidity and disability.

The healthcare system is responsible to apply scientific methods and tools to reduce the impacts of disease and injury, and re-establish health. The development of new methods and tools that help to attain that goal is always desirable. As other scientific fields develop, new possibilities emerge from the combination of different knowledge. This symbiosis inspires the creation of new tools and the introduction of new procedures to improve healthcare.

This work presents a new tool concept which results from the combination of different fields, with the goal of improving the treatment of hospitalised burn patients.

% ==============================
% = The reality of burn injury =
% ==============================
\section{The reality of burn injury} % (fold)
\label{sec:the_reality_of_burn_injury}

Burn injuries are still a prevalent problem in today's society, claiming the lives of about 17 people worldwide, every hour, by 2016 \cite{GHE2016_xls}. Although it represents a small percentage of global death numbers, burn wounds have a significant impact on morbidity and mortality and are considered the worst type of traumatic injury \cite{isbi_guidelines_burn_care}. They also represent a considerable societal cost associated with the expenses for medical care and physical or psychological rehabilitation. When not leading to death, they commonly leave the person "with lifelong disabilities and disfigurements, often with resulting stigma and rejection" \cite{who2011_sucess_stories}.

Injuries result from "an uncontrolled transfer of energy to the human body in amounts that a body cannot tolerate" \cite{who2011_sucess_stories}. Burn injuries are produced by the transfer of radiation or thermal, electrical and chemical energies to the body. Some examples are direct contact with flames, hot objects or liquids (scalds); exposure to ionising radiation; or contact with high electrical currents, or chemicals. 

Various criteria are used to classify burn wounds, being that the mechanism or cause, the extension, and degree or depth, are the most common \cite{who_unicef2008_burns_chapter}. The cause of the wound is related to the energy source involved on the injury formation; It can be further classified as thermal or inhalational, to separate damaged to the skin or to the airways and lungs, respectively. The degree or depth is a very important criteria, because it is directly related to the type of treatment applied. It is separated in three degrees: (a) first-degree, or superficial burns; (b) second-degree, or partial-thickness burns; and (c) third-degree, or full-thickness burns. Finally, the burn extension is clinically referred as the total body surface area (TBSA) burned, or how much surface area of the skin was burned. The most common method used to determine this quantity is called the "rule of nines". This method "assigns 9\% to the head and neck region, 9\% to each arm (including the hand), 18\% to each leg (including the foot) and 18\% to each side of the trunk (back, chest and abdomen)" \cite{who_unicef2008_burns_chapter}. It is only used for adults and children over 10 years. For younger children, the Lund and Browder Chart is used \cite{MacLennan1998_anesthesia_thermal_injury}. 

The TBSA and the wound depth have major impact on morbidity and mortality. Larger TBSA and deeper wounds directly impact the need of emergency care and hospitalisation, the complexity of the treatment, and the length of stay (LoS) {\color{red}(needs ref)}. Naturally, this increases the overall costs of treatment, and the burden of the injury.

\bigskip
According to the World Health Organization (WHO) \cite{who2011_sucess_stories}, the occurrence and outcome severity of burn injuries are higher on Low and Middle Income Countries (LMIC) than on High Income Countries (HIC), with 95\% of annual casualties happening on LMIC. The death rate in LMIC is about 6 times larger than on HIC, reaching around 5.5 deaths/100 000 people per year \cite{who2011_sucess_stories}. 

The disparity on resources between HIC and LMIC means the former has a better leverage to develop strategies to reduce the onset of burn events and lower its morbidity and mortality. Valid strategies consist on: (a) the implementation of effective prevention plans, (b) providing better first-aid and medical care, (c) the application of follow-up intervention for rehabilitation and psychological support, or (d) the development of research efforts to better understand the reality of burn injuries. \cite{who2008_plan_burn_prevention_care, who2011_sucess_stories}. The implementation of these strategies on high income countries have "effectively and sustainably" reduced the number of events and mortality over the last three to four decades \cite{who2011_sucess_stories}. 

Now, the focus of HIC is on better medical care to improve healing and reduce scarring, or to provide rehabilitation and psychological support. Restoring "form and function have become the hallmarks of excellent burn care" \cite{isbi_guidelines_burn_care}. On LMIC, the main focus should be on reducing the occurrence, morbidity and mortality. This means they need to place more effort on prevention, first-aid care and hospital care {\color{red}(needs ref)}.

Epidemiological studies show the distribution of victims of burn injury, according to gender and age. Globally, females represent the majority of the afflicted. "Burns are the only type of unintentional injury where females have a higher rate of injury than males." \cite{who_unicef2008_burns_chapter} Regarding age, infants and elders are the most vulnerable to die from burns; Children represent one fifth off total number of deaths annually {\color{red}(needs ref)}. 

The occurrence by cause of burn varies between HIC and LMIC. In HIC the most common cause are scalds associated with food preparation or consumption and bathing. The victims of these scalds tend to be infants younger than 5 years old. In LMIC flame burns are the most common. They are usually related with the usage of unsafe stoves and lamps. Scalds, generally, lead to non-lethal burns, however, on younger ages, in many cases it leads to death. Burns by flame are more lethal, usually having higher TBSA and depth \cite{who2011_sucess_stories, who_unicef2008_burns_chapter}.

Falar da tendência na europa e em portugal.

\bigskip
The main cause of early death (within the first 24 hours) from burn injury is dehydration and shock caused by a huge loss of body fluids. If the patient survives the first day, the following main cause of death is sepsis. The fragility of the immune system, due to dysfunction of the integumentary system, renders the patients more prone to infectious diseases, such as pneumonia \cite{who2011_sucess_stories}. When a week as passed, the burn site is covered by many organisms, with the most virulent ones starting to invade the healthy tissue \cite{isbi_guidelines_burn_care}. "The avascular nature of the burn predisposes the burn site to bacterial invasion by impeding effective delivery of the antibodies’ own defenses and preventing systemic antibiotics from penetrating the damaged area" \cite{isbi_guidelines_burn_care}. Several types of microbial agents have been found on burn wounds, from bacteria to virus and fungi \cite{Schaal2015a_fungal_infections,Shoja2017_acinetobacter}. If not detected and treated early, these infections can significantly impact morbidity and eventually lead to death.

To prevent microbial infections, burn wounds are first mechanically cleanse by irrigation, followed by topical application of antimicrobial cream and dressing. 



% section the_reality_of_burn_injury (end)

% ===========================================================
% = Development of complementary technologies and practices =
% ===========================================================
\section{Development of complementary technologies and practices} % (fold)
\label{sec:development_of_complementary_technologies_and_practices}

In parallel with the development of medical burn wound care, other areas of science and technology are evolving which can contribute to the improvement of treatment outcome. Three areas are emphasised on this document because they can complement each other to open new roads for treatment or intervention protocols.

The first area, and more directly related with treatment itself, is tissue engineering. The development of this field as brought new perspectives on skin regeneration and healing, which holds promise of significant improvement on burn wound regeneration and scar reduction. 

Secondly, comes computer vision and image processing. The development of faster and more robust algorithms has given computers the capability of automating pattern recognition, allowing them to count cells, detect a person's face, among other feats. Within the context of burn wound care, they can be use to detect the number and size of wounds and estimate the \%TBSA. 

Lastly, collaborative robotics, with new manipulators and control architectures, allows the automation of monotonous and repetitive tasks, leaving the personnel free for more important and demanding tasks. Additionally, it brings a level of precision and accuracy unmatched by humans. Collaborative robotics brings another card to the table by permitting the coexistence of humans and robots on the same workspace.

% section development_of_complementary_technologies_and_practices (end)

% =========
% = Goals =
% =========
\section{Goals} % (fold)
\label{sec:goals}

The present thesis proposes a new system for autonomous topical application of antimicrobial drugs on burn wounds. The main goal of the system is to reduce the time necessary to apply the drugs on extensive wounds, freeing the personnel for other important treatment tasks. This goal encompasses several specific goals, to mention:
\begin{enumerate}
    \item Detection and segmentation of burn wounds;
    \item Definition of wound perimeter and localisation;
    \item Trajectory definition;
    \item Robot manipulator control for trajectory execution;
\end{enumerate}
The broader goal of this work is to prove that there exist other valid use cases, aside from the existing ones, for the use of robotics on surgery and treatment.

% section goals

% ================================
% = Proposed System Architecture =
% ================================
\section{Proposed System Architecture} % (fold)
\label{sec:proposed_system_architecture}

% section proposed_system_architecture

% ======================
% = Document structure =
% ======================
\section{Document structure}
\label{sec:document_structure}

% section document_structure